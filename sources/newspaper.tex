\begin{fancyquote}
	The Chinaman, like men of other races, insists upon indulging in some stimulant or narcotic, and he has chosen opium. He is by no means the teetotaller which he is credited to be. Temperance societies exist in China. Still the Chinaman generally does not indulge in beer or wine --- a great debarrent being the cost when delivered from Europe --- and his \textit{samshu}\footnote{\textit{samshu}, mainly known as \textit{baiju} to natives (\jap{白酒} in Chinese), is a traditional Chinese spirit that is extremely strong at approximately 90 percent proof
	\autocite{Antkiewicz:1993}.} is a weakly subterfuge. The vice which it pleases him to indulge in is, therefore, opium. We have not yet succeeded in introducing temperance, far less abstinence, into England. And you may as soon expect the average Briton to give up his beer or spirits as the Chinaman his pipe. In neither case can you make man moral by Act of Parliament. No reform, I feel certain, is likely to come from the mandarinate, who are nearly without exception slaves to the habit, while the few free from it are powerless against it. It must come, if ever it does, by social reform, from the people themselves. The import of Indian opium by our Government is said by the missionaries to create a source of considerable difficulty in their relations with the Chinese. If not altogether a sincere belief with Chinamen, it is at least a highly convenient argument, and much used by them, that we are largely responsible for the prevalence of the habit; and not only the officials and \textit{literati} [the far-Eastern intelligentsia], but not a few foreigners even, have done their best to foster the idea. True or not true, the charge is one difficult to meet so long as Government preserves its present attitude with regard to Indian opium. Having in view the facts brought forward in this letter --- though of opinion that the suppression of the Indian opium traffic will not stop nor even diminish its use --- I think Government should take steps to discontinue it, and replace it by some other means of revenue. It can hardly be called a creditable source of revenue. From the practical financial as apart from the moral or sentimental aspect, it is advisable to examine the question and seek some means of replacing it. If no such steps be taken, Government will have lost the opportunity of carrying a measure of progress --- an act of self-respect as well as expediency -for the import of Indian opium into China is doomed.
	\begin{flushright}
		\autocite{SpecialCorrespondent:1884}
	\end{flushright}
\end{fancyquote}
