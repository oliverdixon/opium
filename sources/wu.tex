\begin{fancyquote}
        The \textit{Taiping Rebellion}, which lasted from 1850 to 1864,
        represents the type of rebellion which possessed revolutionary
        characteristics. It aimed not only at unseating the Manchus from the
        throne, but sought to establish a new economic and social order.
        Although the \textit{Taiping Rebellion} was finally suppressed, its
        impact upon the existing Manchu order, especially its fiscal system, was
        great. The Manchu government was forced by the rebellion into a series
        of changes and reforms which fundamentally altered the power structure
        of China and paved the way for the revolutionary movements of the
        following century. [\ldots]

        As a result, there was conflict between gentry and
        magistrate\footnote{The conflict that \textit{Wu} is referring to
        concerns the fact that local officials were free, under the
        newly-created decentralised system, to collect revenues beyond the
        minimum quotas set by the Government in Peking. This allowed the
        officialdom to exploit the masses and use collected funds for their own
        personal gains as opposed to their intended uses of fixing various
        elements of communities such as river dams and acting as ``safety nets''
        for natural events such as food shortages and droughts. Subsequently,
        many communities began to fall into dire straits.}. This chaos was
        increased by the rapacity of public-works intendants who pocketed funds
        intended for the maintenance of rivers and dams. Hence dykes went into
        disrepair, rivers became clogged, and an aftermath of floods and
        droughts brought constant and increasing famine. [\ldots]
	\begin{flushright}
		\autocite{Wu:1950}
	\end{flushright}
\end{fancyquote}

