\begin{fancyquote}
        [\ldots] The Western impact on China from the 1880s onwards was entirely
        different from that which preceded it. It did not consist of gunboat
        diplomacy of imported foreign goods, such as Indian opium or British
        machine-made cotton goods, let alone advanced Western civilisation.
        Previous studies, which supposed that `Western impact' was the
        above-mentioned were based on a false assumption that only Westeners
        could produce a Western impact on China.

        In fact, the Chinese themselves were capable of promoting a Western
        impact on China. The crucial change in treaty-port society in China
        during the 1880s was the emerge of a new type of Chinese people who
        could speak and write English fluently and were familiar Western
        culture. According to [an] article in the \textit{North-China Herald},
        Western residents in China did not entirely trust and welcome these
        `English-speaking Chinese', due to their ambiguous
        character\footnote{The term ``ambiguous character'' can be used to
        intensify the extents to which both the British and Chinese felt their
        respective cultures were wholly incompatible, and that the absurd
        proposition of an ``English-speaking Chinese'' was not true to either
        culture. \autocite{Lyczak:1976}}. [\ldots]

        So, according to my new definition, `the Western impact on China' after
        the 1880s consisted of the arrival of Western institutions stipulated by
        the `unequal' treaties and the growth of commercial networks of
        English-speaking Chinese who cooperated with foreign firms in order to
        benefit from the treaty privileges. With these privileges, the
        English-speaking Chinese and foreign firms could develop their
        commercial networks. [\ldots]

        Why could the commercial activities of the English-speaking Chinese
        undermine the apparently unshakable solidarity of the prominent Chinese
        merchants' groups and thus threaten the control the Qing government
        officials exercised over them, especially from the latter half of the
        1890s onwards? [\ldots]
	\begin{flushright}
		\autocite{Motono:2000}
	\end{flushright}
\end{fancyquote}

