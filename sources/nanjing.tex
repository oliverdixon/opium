\begin{mdframed}[style=fancyquote]
	\articlehead{[\ldots] Article II}
        \textit{His Majesty the Emperor of China} agrees, that British Subjects,
        with their families and establishments, shall be allowed to reside, for
        the purpose of carrying on their mercantile pursuits, without
        molestation or restraint, at the cities and towns of Canton, Amoy,
        Foochowfoo, Ningpo, and Shanghai; and \textit{Her Majesty the Queen of
        Great Britain}, will appoint Superintendents, or Consular Officers, to
        reside at each of the above named Cities, or Towns, to be the medium of
        communication between the Chinese Authorities, and the said merchants,
        and to see that the just Duties and other Dues of the Chinese
        Government, as hereafter provided for, are duly discharged by
        \textit{Her Britannic Majesty}'s subjects.

        \vspace*{1em}
        \articlehead{Article III [\ldots]}
        It being obviously necessary and desirable, that British Subjects should
        have some Port whereat they may careen and refit their Ships, when
        required, and Keep Stores for that purpose, \textit{His Majesty the
        Emperor of China} cedes to \textit{Her Majesty the Queen of Great
        Britain}, the Island of Hong-Kong, to be possessed in perpetuity by
        \textit{Her Britannic Majesty}, Her Heirs and Successors, and to be
        governed by such Laws and Regulations as \textit{Her Majesty the Queen
        of Great Britain}, shall see fit to direct.

        \vspace*{1em}
        \articlehead{[\ldots] Article VII [\ldots]}
        It is agreed that the Total amount of Twenty One Millions of Dollars,
        described in the three preceding Articles, shall be paid as follows: six
        millions immediately [\ldots]; six millions in 1843; five millions in
        1844; four millions in 1845. It is further stipulated, that Interest, at
        the rate of five per cent per annum, shall be paid by the Government of
        China on any portions of the above sums that are not punctually
        discharged at the periods fixed.
	\begin{flushright}
		\autocite{Mayers:1902}
	\end{flushright}
\end{mdframed}
