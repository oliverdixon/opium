\textrule
\vspace{0.6em}
{\centering \textsc{[\ldots] Article II} \\[1em]}
\textit{His Majesty the Emperor of China agrees, that British Subjects, with their families and establishments, shall be allowed to reside, for the purpose of carrying on their mercantile pursuits, without molestation or restraint, at the cities and towns of Canton, Amoy, Foochowfoo, Ningpo, and Shanghai; and Her Majesty the Queen of Great Britain, will appoint Superintendents, or Consular Officers, to reside at each of the above named Cities, or Towns, to be the medium of communication between the Chinese Authorities, and the said merchants, and to see that the just Duties and other Dues of the Chinese Government, as hereafter provided for, are duly discharged by Her Britannic Majesty's subjects.}

{\centering \textsc{Article III [\ldots]} \\[1em]}
\textit{It being obviously necessary and desirable, that British Subjects should have some Port whereat they may careen and refit their Ships, when required, and Keep Stores for that purpose, His Majesty the Emperor of China cedes to Her Majesty the Queen of Great Britain, the Island of Hong-Kong, to be possessed in perpetuity by Her Britannic Majesty, Her Heirs and Successors, and to be governed by such Laws and Regulations as Her Majesty the Queen of Great Britain, shall see fit to direct.} \\

{\centering \textsc{[\ldots] Article VII [\ldots]}\footnote{This article has been edited for the sake of brevity, however all information remains accurate.} \\[1em]}
\textit{It is agreed that the Total amount of Twenty One Millions of Dollars, described in the three preceding Articles, shall be paid as follows: six millions immediately \emph{[\ldots]}; six millions in 1843; five millions in 1844; four millions in 1845. It is further stipulated, that Interest, at the rate of five per cent per annum, shall be paid by the Government of China on any portions of the above sums that are not punctually discharged at the periods fixed.} \\
\textrule \\
