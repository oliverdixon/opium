% 鴉片戰爭 The Opium Wars, Oliver Dixon
% History Coursework Dissertation LaTeX source (designed for PDFLaTeX)

\documentclass{article}

\usepackage[a4paper, total={6in, 10in}]{geometry}
\usepackage[bottom]{footmisc} % make footnotes look like footnotes
\usepackage{CJKutf8,pinyin,multicol,graphicx}

\usepackage[
	backend=biber,
	sorting=none,
	bibencoding=utf8,
	style=alphabetic
]{biblatex}

\addbibresource{main.bib}
\defbibheading{bibliography}{\section*{Cited Works}}

\usepackage{hyperref}
\hypersetup {
	colorlinks = true,
	allcolors = {blue}
}

\setlength{\footnotesep}{1.5em}

% for displaying Chinese text (\normalfont is used to prevent a warning from CJK)
\newcommand{\zh}[1]{\begin{CJK}{UTF8}{bsmi}\normalfont #1\end{CJK}} % 繁體 Trad. Chinese
\newcommand{\zhs}[1]{\begin{CJK}{UTF8}{gbsn}\normalfont #1\end{CJK}} % 簡體(简体)Simp. Chinese

% thin horizontal line spanning the width of text
\newcommand{\textrule}{\noindent\makebox[\linewidth]{\rule{\linewidth}{0.4pt}}}

% fancyquote: \fancyquote{<quote text>}{<author(s)>}{<date>}{<citation>}
\newcommand{\fancyquote}[4]{
	\begin{quote}
		\textrule #1
		\begin{flushright}
			\textit{#2 (#3)} #4 \textrule
		\end{flushright}
	\end{quote}
}

\title{``\emph{The Opium Wars were the Primary Reason for the Downfall of the Qing Dynasty in the Years 1814-1912}'' --- A Historical Analysis}
\author{Oliver Dixon}
\date{Summer 2019}

\begin{document}

\maketitle

\begin{abstract}

	Succeeding the Ming dynasty, the great Qing dynasty ruled from 1644 to 1912, achieving the most significant strides ever witnessed on the Chinese political spectrum. Aside from doubling the land under the empire \autocite{Turchin:2006} while tripling the population through encapsulation of a broad range\footnote{Such ethnicities include the Uyghur Muslims native to Xinjiang, Tibetans, Mongols, and Burmese.} of ethnicities within the Chinese state \autocite{Rowe:2012}, the dynasty enjoyed economic superiority of proportions that Europe, the Americas, and their other Asian neighbours could only imagine \autocite{Maddison:2007}. Despite their initial prosperity and success as an irreplaceable and integral element of the rapidly-increasing worldwide economy, the Qing empire suffered the fate of been the final imperial dynasty of China due to an array of factors, both international and domestic.

	This essay will discuss the factors which led to the downfall of the greatest dynasty in the history of China. It will also examine the extent to which the strained international trade relations with the East India Company and the ensuing Opium wars contributed to the events of 1912.

\end{abstract}

\textrule
\vspace*{0.6em}
\centering
\textsc{This ``non-examined assessment'' document is presented as partial fulfilment of the \textit{A.Q.A. A-Level History} qualification} \\
\textrule

\begin{multicols}{2}


\end{multicols}

\printbibliography

\end{document}
