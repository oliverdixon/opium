% The Opium Wars, Oliver Dixon
% History Coursework Dissertation LaTeX source (designed for PDFLaTeX)

\documentclass{article}

\usepackage[a4paper, total={6in, 10in}]{geometry}
\usepackage[bottom]{footmisc} % make footnotes look like footnotes
\usepackage[super]{nth}

\usepackage[
	backend=biber,
	sorting=none,
	bibencoding=utf8,
	style=alphabetic
]{biblatex}

\addbibresource{main.bib}

\usepackage{hyperref}
\hypersetup {
	colorlinks = true,
	allcolors = {blue}
}

\setlength{\footnotesep}{1em}

% thin horizontal line spanning the width of text
\newcommand{\textrule}{\noindent\makebox[\linewidth]{\rule{\linewidth}{0.4pt}}}

\title{``\emph{The Opium Wars were the Primary Reason for the Downfall of the Qing Dynasty in the Years 1814-1912}'' --- A Historical Analysis}
\author{Oliver Dixon}
\date{Summer 2019}

\begin{document}

\maketitle

\begin{abstract}

	Succeeding the Ming dynasty, the great Qing dynasty ruled from 1644 to 1912, achieving the most significant strides ever witnessed on the Chinese political spectrum. Aside from doubling the land under the empire
	\autocite{Turchin:2006} while tripling the population through encapsulation of a broad range\footnote{Such ethnicities include the Uyghur Muslims native to Xinjiang, Tibetans, Mongols, and Burmese.} of ethnicities within the Chinese state
	\autocite{Rowe:2012}, the dynasty enjoyed economic superiority of proportions that Europe, the Americas, and their other Asian neighbours could only imagine 
	\autocite{Maddison:2007}. Despite their initial prosperity and success as an irreplaceable and integral element of the rapidly-increasing worldwide economy, the Qing empire suffered the fate of taking the place as the final imperial dynasty of China due to an array of factors, both international and domestic.

	This essay will discuss the factors which led to the downfall of the greatest dynasty in the history of China. Also under examination will be the extent to which the strained international trade relations with the East India Company and the ensuing Opium wars contributed to the ultimate demise of 1912.

\end{abstract}

\textrule
\vspace*{0.6em}
{\centering \textsc{This ``Non-Examined Assessment'' Document is Presented as Partial Fulfilment of the \textit{A.Q.A. A-Level History} Qualification}\\}
\textrule

\section{Introduction}

From the exposure of the weaknesses in the Qing military to the anti-imperialist and anti-feudalism domestic rebellions, it is clear that the Opium wars only ever existed as a subset of the immediate issues threatening the Great dynasty. The ever-diminishing national identity in the face of foreign intervention is undoubtedly a significant factor when inspecting the ultimate demise of China's final dynasty, however the somewhat-inadvertent internationalisation efforts of the global powerhouse cannot be solely attributed to the onset of the Opium wars.

Despite not existing as a sole catalyst for the Qing downfall, the Opium wars bore such significance\footnote{Historians W. Travis Hanes III, Ph.D. and Frank Sanello communicate the magnitude of the Opium wars by comparing the nineteenth-century events to the modern-day U.S. government legalising cocaine and allowing its uncontrolled trade succeeding a defeat in a military offensive launched by the cocaine cartels of Columbia
\autocite{Hanes:2004}.} in the making of modern China that it would be improper to suggest they were anything other than the primary causes.

The first Opium war initiated a state of mass-unrest throughout the region, invoking what some have called a ``\textit{Chinese people's bourgeois-democratic revolution against imperialism and feudalism}''
\autocite{Janin:1999}, which is incredibly threatening due to its direct contradiction with the principles and philosophies of the imperialist Qing. Various wars fought between the Qing empire and Western powers ultimately led to what are now referred to as the ``Unequal Treaties'', in which the Qing was diplomatically forced to sign a series of unfavourable agreements with the Western powers \autocite{Wang:2005}.

Additionally, some have argued that the importation of opium was not directly attributable for the downfall of the empire, but rather the Boxer rebellion and subsequent instantiation of the Eight-Nation alliance, which is an indirect consequence. Historians whom argue this perspective largely cite the ``Boxer Protocol''. Often considered to be included in the series of Unequal Treaties imposed upon the Qing, \footnote{The nations involved in the anti-Qing alliance were Japan, Britain, France, the United States, Russia, Germany, Italy, and Austria-Hungary. They were also supported by Australian and Indian forces of the British empire
\autocite{Gardener:2016}.} the protocol was the driving force which ultimately ``[brought] the Great dynasty to its knees'' due to crippling debt \autocite{Mitchell:2008}.

However, it would also be improper to avoid considering the events which were not directly attributable to the West, shifting the focus to China's eastern neighbours. Since their ascension to power in 1644, the Qing dynasty has attempted to incorporate various non-Manchurian populations into the China proper, leading to an inevitably-factionalised society. Extending to the military of the Qing, the 240,000-strong Japanese armies brutally humiliated the 600,000-strong Chinese forces during the first Sino-Japanese war
\autocite{Fenby:2013}, exposing the widespread incompetence and lack of training possessed by the Sino armed forces
\autocite{Jowett:2013} when fighting for influence in the Korean peninsula.

\section{Analysis of Primary Sources}

\subsection{The Treaty of Nanjing (Nanking)}

The \textit{Treaty of Nanjing}\footnote{The treaty is less-commonly known by its official name of \textit{Treaty of Peace, Friendship, and Commerce Between Her Majesty the Queen of Great Britain and Ireland and the Emperor of China}.}, also known as the \textit{Treaty of Nanking}\footnote{Due to the vast differences of the Chinese and English languages, various \textit{romanisation} methods were devised. ``\textit{Nanking}'' is the \textit{Wade-Giles} romanisation, whereas ``\textit{Nanjing}'' is the increasingly-popular \textit{Pinyin} romanisation. See \url{https://web.archive.org/web/20140714185035/http://www.sacu.org/roman.html} [archived] for more information.}, was a treaty between Britain and the Qing to end the \textit{First Opium War}. The general consensus regarding the treaty states that despite its undebatable harshness and inequality, the Qing representative (Qingyi) had no choice but to sign the agreement due to the their status as a ``defeated nation''
\autocite{Mao:2018}. The following extract is from the thirteen-article treaty, highlighting the most significant elements
\autocite{Mayers:1902}.

\textrule
\vspace{0.6em}
{\centering \textsc{[\ldots] Article II} \\[1em]}
\textit{His Majesty the Emperor of China agrees, that British Subjects, with their families and establishments, shall be allowed to reside, for the purpose of carrying on their mercantile pursuits, without molestation or restraint, at the cities and towns of Canton, Amoy, Foochowfoo, Ningpo, and Shanghai; and Her Majesty the Queen of Great Britain, will appoint Superintendents, or Consular Officers, to reside at each of the above named Cities, or Towns, to be the medium of communication between the Chinese Authorities, and the said merchants, and to see that the just Duties and other Dues of the Chinese Government, as hereafter provided for, are duly discharged by Her Britannic Majesty's subjects.}

{\centering \textsc{Article III [\ldots]} \\[1em]}
\textit{It being obviously necessary and desirable, that British Subjects should have some Port whereat they may careen and refit their Ships, when required, and Keep Stores for that purpose, His Majesty the Emperor of China cedes to Her Majesty the Queen of Great Britain, the Island of Hong-Kong, to be possessed in perpetuity by Her Britannic Majesty, Her Heirs and Successors, and to be governed by such Laws and Regulations as Her Majesty the Queen of Great Britain, shall see fit to direct.} \\

{\centering \textsc{[\ldots] Article VII [\ldots]}\footnote{This article has been edited for the sake of brevity, however all information remains accurate.} \\[1em]}
\textit{It is agreed that the Total amount of Twenty One Millions of Dollars, described in the three preceding Articles, shall be paid as follows: six millions immediately \emph{[\ldots]}; six millions in 1843; five millions in 1844; four millions in 1845. It is further stipulated, that Interest, at the rate of five per cent per annum, shall be paid by the Government of China on any portions of the above sums that are not punctually discharged at the periods fixed.} \\[0.6em]
\textrule \\

The treaty, signed \nth{29} August 1842 to come into effect on \nth{26} June 1843
\autocite{Wright:2007}, possesses unparalleled potential usage when conducting an investigation into the causes of the downfall of the Qing empire. Despite its the fact that its implementation precedes the events of 1912 by seventy years, the effects of these articles represent the first significant international tension which China endured, causing some scholars to argue that the treaty was the initiator and catalyst for the downfall. The three selected articles exasperate the true strain which was placed on the empire; not only losing control of Hong Kong to the British crown\footnote{Somewhat-entertainingly, the clause in the Chinese text euphemistically states that ``the emperor graciously grants a place of rest and storage to the British after their long voyage to China''
\autocite{Zhang:2007}.}, but also the forfeiture of \$21,000,000 to \textit{Her Majesty}\footnote{The penalty of \$21 million to \textit{Her Majesty} was broken down as follows: \$12 million for military expenses, \$6 million for the opium which China had destroyed, and \$3 million for the repayment of Chinese merchants' debts to the British
\autocite{Hsu:1999}.} in addition to the free movement of British merchants throughout multiple Chinese ports on the mainland\footnote{As per the treaty text, these ports are \textit{Canton}, \textit{Amoy}, \textit{Foochowfoo}, \textit{Ningpo}, and \textit{Shanghai}.}. As Britain suffered no obligations in return, the treaty became known as the first of the \textit{Unequal Treaties} signed between the Qing and Britain
\autocite{Hoe:1999}. Some scholars also state that as ``Missionaries could preach within the empire [...], literary undertakings rather gave way to chapels, schools, and hospitals''
\autocite{Britton:1933}, implicitly suggesting that the opening of Hong Kong and the five ports allowed the empire to prosper under British rule, hence undermining the authority and respect which had previously been owned solely by the Qing emperor\footnote{An exemplar of the ways in which Hong Kong prospered is the fact that during the period 1841-60, ``the number of English and Chinese periodicals and newspapers published in Hong Kong was more than the total number in the rest of China''
\autocite{Huang:2001}.}.

According to
\autocite{Hsu:1999}, ``[the] treaty was imposed by the victor upon the vanished at gunpoint, without the careful deliberation usually accompanying international agreements in Europe and America''. This comment on provenance increases the value of the source beyond its verbatim content, as the circumstances in which it was signed, where China were referred to as the ``vanished [nation]'', amplify its use into the realms of rendering the once Great dynasty as a completely-powerless and helpless state at the whim of the British.

\subsection{A Letter of Advice from Lin Zixu Lin Tse-Hs\"u to Queen Victoria}

\autocite{Teng:1995}

\pagebreak
\printbibliography[title={Cited Works}, heading=bibintoc]

\end{document}
