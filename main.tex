% "The Opium Wars were the Primary Reason for the Downfall of the Qing Dynasty
% in the Years 1814-1912" Oliver Dixon, History Coursework Dissertation LaTeX
% source (designed for PDFLaTeX)

\documentclass[a4paper,oneside]{article}

\usepackage[a4paper, total={6in, 10in}]{geometry}
\usepackage[bottom]{footmisc} % make footnotes look like footnotes
\usepackage[super]{nth} % place "st", "nd", "rd", "th", etc. in the superscript
\usepackage{graphicx,mdframed,xcolor} % frame environments and Chinese decor
\usepackage{indentfirst} % indent new paragraphs
\usepackage{CJKutf8} % Japanese and Chinese text (see \jap, \zht, and \zhs)

% --- WORD COUNTS --- %

\newcommand{\wordcount}{3649}
\newcommand{\wordcountabstract}{3627}

% --- BIBLIOGRAPHY SET-UP --- %

\usepackage[
	backend=biber,
	sorting=none,
	bibencoding=utf8,
	style=alphabetic
]{biblatex}

\addbibresource{main.bib}

% Break long U.R.L.s which do not contain breaking characters
% -- from https://tex.stackexchange.com/a/134281
\setcounter{biburllcpenalty}{7000}
\setcounter{biburlucpenalty}{8000}

\newcommand{\jap}[1]{% Japanese
        \begin{CJK*}{UTF8}{min}%
                \normalfont%
                #1
        \end{CJK*}
}

\newcommand{\zht}[1]{% Traditional Chinese
        \begin{CJK*}{UTF8}{bsmi}%
                \normalfont%
                #1
        \end{CJK*}
}

\newcommand{\zhs}[1]{% Simplified Chinese
        \begin{CJK*}{UTF8}{gbsn}%
                \normalfont%
                #1
        \end{CJK*}
}

% --- HYPERREF IMPORT AND SET-UP --- %

\usepackage{hyperref}
\hypersetup{%
	colorlinks = true,
	allcolors = {blue}
}

% --- ENVIRONMENT AND COMMAND SET-UP --- %

% thin horizontal line spanning the width of text
\newcommand{\textrule}{%
        \noindent
        \makebox[\linewidth]{
                \rule{\linewidth}{0.4pt}
        }%
}

% intended for printing headers for multiple-heading sources
\newcommand{\articlehead}[1]{
        \noindent
        \qquad
        \textbf{#1}
        \qquad
}

\mdfdefinestyle{fancyquote}{linecolor=lightgray}

% \tolerance=10000 % "infinite" tolerance (see TeXBook page 29)
\emergencystretch=1em
\setlength{\footnotesep}{1em}

% --- FANCYHDR SET-UP --- %

\usepackage{fancyhdr,titling}

\fancypagestyle{stdhdr}{%
	\fancyhf{}
	\fancyhead[R]{\theauthor}
        \fancyhead[L]{\textit{A.Q.A. A-Level History} Non-Examined Assessment}
        \fancyfoot[L]{\textbf{Draft Copy --- \today}}
	\fancyfoot[R]{Page \thepage}
}

\fancypagestyle{titlehdr}{%
	\fancyhead{}
	\renewcommand{\headrulewidth}{0pt}
}

\title{``\emph{The Opium Wars were the Primary Reason for the Downfall of the
        Qing Dynasty in the Years 1814 to 1912}'' --- A Historical Analysis}
\author{Oliver Dixon}
\date{Summer, 2019}

% ---------------------- %
% --- DOCUMENT START --- %
% ---------------------- %

\begin{document}

\clearpage\maketitle
\thispagestyle{titlehdr}
\pagestyle{stdhdr}

\vspace*{-1.5em}
\begin{figure}[h!]
	\centering
        \def\svgwidth{0.5\linewidth}
	\input{chinese_frontcover.pdf_tex}
\end{figure}
\vspace*{-0.5em}

\begin{abstract}

        Succeeding the Ming dynasty, the great Qing dynasty ruled from 1644 to
        1912, achieving the most significant strides ever witnessed on the
        Chinese political spectrum. Aside from doubling the land under the
        empire \autocites{Turchin:2006}{Goldstone:1995} whilst tripling the
        population through encapsulation of a broad range\footnote{Such
        ethnicities include the Uyghur Muslims native to Xinjiang, Tibetans,
        Mongols, and Burmese \autocite{Chia:1993}. These ethnic groups, often
        rather reluctantly \autocites{Teichman:2002}{Smith:2009}{Dwyer:2005},
        still remain part of China today. However, it is also important to note
        that although these ethnic groups were ``encapsulated'', they were by no
        means assimilated.  This view was originally proposed by \textit{Rawski}
        in her 1996 thesis \autocite{Rawski:1996}, however is now a commonplace
        opinion \autocite{Hou:2014}.} of ethnicities within the Chinese state
        \autocite{Rowe:2012}, the dynasty enjoyed economic superiority of
        proportions that Europe, the Americas, and their other Asian neighbours
        could only imagine \autocite{Maddison:2007}. Despite their initial
        prosperity and success as an irreplaceable and integral element of the
        rapidly-increasing worldwide economy, the Qing empire suffered the fate
        of taking the place as the final imperial dynasty of China due to an
        array of factors, both international and domestic.

        Towards the end of the dynasty, emergency reforms were passed in
        fruitless attempts of strengthening the China proper by means of
        unification, such as lifting the ban on the marriages of Manchurians and
        Han Chinese in 1902 \autocite{Rhoads:2000}, however the recent intrusion
        of foreign powers into Beijing ensured the absolute end of imperialist
        China.

        This essay will discuss the factors which led to the downfall of the
        greatest dynasty\footnote{It has been repeatedly suggested, by the
        \textit{Naito} hypothesis, that the emperor of China became more secure
        after the Song dynasty \autocite{Miyakawa:1955}: ``Usurpation became
        rare after the introduction of examinations to select officials.''
        \autocite{Sng:2014}} in the history of China. Also under examination
        will be the extent to which the strained international trade relations
        with the \textit{East India Company} and the ensuing Opium wars
        contributed to the ultimate demise of 1912.

\end{abstract}

\textrule%
\vspace*{0.6em}
{
        \centering \textsc{This ``Non-Examined Assessment'' Document is
        Presented as Partial Fulfilment of the \textit{A.Q.A. A-Level History}
        Qualification}\\
}
\textrule%
\vspace*{0.6em}

From the exposure of the weaknesses in the Qing military to the
anti-imperialist and anti-feudalism domestic rebellions, it is clear that the
Opium wars only ever existed as a subset of the immediate issues threatening
the Great dynasty.  Despite \textit{Emperor Jiaqing} issuing an anti-opium
edict\footnote{A small extract from \textit{Jiaqing}'s condemnation of the
\textit{East India Company} reads ``As for opium, it is poisonous to the
greatest extreme, and those who have taken it are all wicked and unscrupulous
without exception''. Speaking in 1814, it is clear that his words had little
effect on the actions of the traders.} in 1814 \autocite{Hu:1991}, the
ever-diminishing national identity\footnote{According to contemporary British
travellers to China, a true Chinaman often despised all other nations than his
own, while considering foreigners to be ``barbarians''
\autocite{McPherson:1842}.  These notes were published in the midst of the
\textit{First Opium War} however, hence the general animosity towards
foreigners was likely to have been understandably at its peak during the
traveller's expedition.} in the face of foreign intervention is undoubtedly a
significant factor when inspecting the ultimate demise of China's final
dynasty, however the somewhat-inadvertent internationalisation efforts of the
global powerhouse cannot be solely attributed to the onset of the opium wars.

Despite not existing as a sole catalyst for the Qing downfall, the opium wars
bore such significance\footnote{Historians \textit{William Travis Hanes III},
Ph.D. and \textit{Frank Sanello} communicate the magnitude of the opium wars by
comparing the nineteenth-century events to the modern-day U.S.\ government
legalising cocaine and allowing its uncontrolled trade succeeding a defeat in a
military offensive launched by the cocaine cartels of Columbia
\autocite{Hanes:2004}.} in the making of modern China that it would be improper
to suggest they were anything other than the primary causes.

The \textit{First Opium War} initiated a state of mass-unrest throughout the
region, invoking what some have called a ``Chinese people's bourgeois-democratic
revolution against imperialism and feudalism'' \autocite{Janin:1999}, which is
incredibly threatening due to its direct contradiction with the principles and
philosophies of the imperialist Qing. Various wars fought between the Qing
empire and Western powers ultimately led to what are now referred to as the
\textit{Unequal Treaties}, in which the Qing was diplomatically forced to sign a
series of unfavourable agreements with the Western powers \autocite{Wang:2005}.

Additionally, some have argued that the importation of opium and ensuing wars
were not directly attributable for the downfall of the empire, but rather the
\textit{Boxer Rebellion} (1899 to 1901) and subsequent instantiation of the
\textit{Eight-Nation Alliance}, which is an indirect consequence. Historians
arguing this perspective largely cite the \textit{Boxer Protocol}. Often
considered to be included in the series of \textit{Unequal Treaties} imposed
upon the Qing\footnote{The nations involved in the anti-Qing alliance were
Japan, Britain, France, the United States, Russia, Germany, Italy, and
Austria-Hungary. They were also supported by Australian and Indian forces of the
British empire \autocite{Gardener:2016}.}, the protocol was the driving force
which ultimately ``[brought] the Great dynasty to its knees'' due to crippling
debt \autocite{Mitchell:2008}.

However, it would also be improper to avoid considering the events which were
not directly attributable to the West, shifting the focus to China's eastern
neighbours. Since their ascension to power in 1644, the Qing dynasty attempted
to incorporate various non-Manchurian populations into the China proper, leading
to an inevitably-factionalised society, such that many Manchu households, also
known as `banner' households, were facing extreme poverty from the 1820s
\autocite{Elliott:2006}.  Extending to the military of the Qing, the
240,000-strong Japanese armies brutally humiliated the 600,000-strong Chinese
forces during the \textit{First Sino-Japanese War} (1894 to 1895)
\autocite{Fenby:2013}, exposing the widespread incompetence and lack of training
possessed by the Sino armed forces \autocite{Jowett:2013} when fighting for
influence in the Korean peninsula. An example of the Qing military's
incompetence can be shown by the fact that many weapons used were relics from
the early Qing period and even some from the Ming period \autocite{Qi:1964}.

Additionally relevant, and a point of such interest that it will be examined in
detail during this essay, are the effects of the \textit{Taiping Rebellion}
(1850 to 1864) upon the Manchurian-imposed societal structures, reducing the
once-great empire to a decentralised `fiasco' facing a revolution formed by a
man making some incredibly dubious claims about his ancestry.

A useful piece of contemporary evidence in support of the outlined argument can
be found from \textit{Lin Tse-Hs\"u}, also known as \textit{Lin Zexu}.
\textit{Lin} was the Chinese commissioner in Canton, sent by the Qing
emperor\footnote{The Qing emperor, \textit{Daoguang}, allegedly lost his son,
\textit{Xianfeng}, to an opium-induced overdose in 1861, possibly fuelling his
absolute hatred of the opium trade \autocite{Ringmar:2013}. \textit{Xianfeng}'s
death was, as expected, incredibly controversial with anti- and pro-opium
activists alike, submitting that the hypocrisy of his acts discredit all
anti-opium opinions of the Qing. Before his death, he presented with massive
haemoptysis: the ``coughing-up'' of blood in excess of 600 millilitres
\autocite{Sabatine:2013}.} to negotiate an end to the importation of opium by
the \textit{East India Company} \autocite{ChinaHistoricalArchives:1992}. In
1839, preceding the \textit{First Opium War} by three years, \textit{Lin} wrote
a strongly-worded letter of advice to \textit{Queen Victoria}, stating his
strong desire to end the importation of opium by any means necessary.

\begin{fancyquote}
	\emph{[\ldots]} But after a long period of commercial intercourse, there appear among the crowh of barbarians both good persons and bad, unevenly. Consequently there are those who smuggle opium to seduce the Chinese people and so cause the spread of the poison to all provinces. Such persons who only care to profit themselves, and disregard their harm to others, are not tolerated by the laws of heaven and are unanimoly hated by human beings. His Majesty the Emperor, upon hearing of this, is in a towering rage. He has especially sent me, his commissioner, to come to Kwangtung, and together with the governor-general and governor jointly to investigate and settle this matter.\par
	All those people in China who sell opium or smoke opium should receive the death penalty. We trace the crime of those barbarians who through the years have been selling opium, then the deep harm they have wrought and the great profit they have usurped should fundamentally justify their execution according to law. We take into to consideration, however, the fact that the various barbarians have still known how to repent their crimes and return to their allegiance to us by taking the 20,183 chests of opium from their storeships and petitioning us, through their consular officer [superintendent of trade], Elliot, to receive it. It has been entirely destroyed and this has been faithfully reported to the Throne in several memorials by this comissioner and his colleagues. \emph{[\ldots]}
	\begin{flushright}
		\emph{\autocite{Teng:1979}}
	\end{flushright}
\end{fancyquote}
 % LETTER TO QUEEN VICTORIA, TeX-EMBED

This letter is an extremely useful piece of evidence when conducting an
investigation into any aspect of the Qing dynasty, as it is written with an
extraordinarily strong sense of foreshadowing for a violent
conflict\footnote{Some scholars have stated that this letter is a somewhat
unique piece of attempted diplomacy from China, as the incredibly direct letter
violates their usual conventions of writing in a ``highly-stylised'' fashion
\autocite{Kishlansky:1995}. This is a potential indicator to the true severity
of the situation, such that a well-established governmental Chinaman would have
correspondence in such a personalised and honest manner.} between the anti-opium
groups and the \textit{East India Company}. It becomes increasingly valuable
when considering the author, recipient, and the circumstances in which the
letter was written. Sent to \textit{Queen Victoria} by the commissioner of the
primary trading port of the \textit{East India Company}, \textit{Lin} was likely
in a more hopeful position to make a plea to \textit{Her Majesty} than any other
entity in China due to his familiarity and knowledge on the matter. His extreme
stance on serving the death penalty to distributors and users of opium
represents the severity of the situation throughout China, while also
exasperating the level to which opium-smoking was widespread to ``all
provinces''.

The unfathomably widespread nature of opium throughout all Chinese provinces
renders \textit{Lin}'s strongly-worded advice as an inescapable
point-of-interest, as it undisputedly provides evidence to show the fundamental
scar that opium had left on the Chinese population, and the desperation of the
Qing to restore order by any means necessary.

This source could be somewhat-limited due to the fact it was written by a
high-ranking member of the Qing dynasty, having a strong inclination to
exaggerate the effects of British presence in China in order to appease the
Emperor, however the majority of his writings are known to be factual to a
rather striking degree due to the extensive number of other contemporary records
concerning the severity of opium.

In this particular case of attempted diplomacy, it is a matter of debate as to
whether \textit{Queen Victoria} ever read the letter.

Moving towards documents with greater formality, the level of which the opium
wars affected the Qing remain startlingly apparent. Often hailed as a
``turning-point in far-Eastern history'' \autocite{Fairbank:1940}, the
\textit{Treaty of Nanjing} (1842), also known as the \textit{Treaty of
Nanking}\footnote{Due to the vast differences of the Chinese and English
languages, various romanisation methods were devised.  ``Nanking'' is the
\textit{Wade-Giles} romanisation, whereas ``Nanjing'' is the increasingly
popular \textit{Pinyin} romanisation, the latter of which has been commonplace
in both governmental and non-governmental documents since the 1970s
\autocite{Tao:1991}. Some Chinese names remain in the \textit{Wade-Giles} style
even in modern texts however, as their occurrence has become popular in the
West.}, was a treaty between Britain and the Qing to end the \textit{First Opium
War}. The general consensus regarding the treaty states that despite its
undebatable harshness and inequality, the Qing representative (\textit{Qingyi})
had no choice but to sign the agreement due to the their status as a ``defeated
nation'' \autocite{Mao:2018}. The following extract is from the thirteen-article
treaty, highlighting the most significant elements.

\begin{mdframed}[style=fancyquote]
	\articlehead{[\ldots] Article II}
        \textit{His Majesty the Emperor of China} agrees, that British Subjects,
        with their families and establishments, shall be allowed to reside, for
        the purpose of carrying on their mercantile pursuits, without
        molestation or restraint, at the cities and towns of Canton, Amoy,
        Foochowfoo, Ningpo, and Shanghai; and \textit{Her Majesty the Queen of
        Great Britain}, will appoint Superintendents, or Consular Officers, to
        reside at each of the above named Cities, or Towns, to be the medium of
        communication between the Chinese Authorities, and the said merchants,
        and to see that the just Duties and other Dues of the Chinese
        Government, as hereafter provided for, are duly discharged by
        \textit{Her Britannic Majesty}'s subjects.

        \vspace*{1em}
        \articlehead{Article III [\ldots]}
        It being obviously necessary and desirable, that British Subjects should
        have some Port whereat they may careen and refit their Ships, when
        required, and Keep Stores for that purpose, \textit{His Majesty the
        Emperor of China} cedes to \textit{Her Majesty the Queen of Great
        Britain}, the Island of Hong-Kong, to be possessed in perpetuity by
        \textit{Her Britannic Majesty}, Her Heirs and Successors, and to be
        governed by such Laws and Regulations as \textit{Her Majesty the Queen
        of Great Britain}, shall see fit to direct.

        \pagebreak % likely temporary
        \articlehead{[\ldots] Article VII [\ldots]}
        It is agreed that the Total amount of Twenty One Millions of Dollars,
        described in the three preceding Articles, shall be paid as follows: six
        millions immediately [\ldots]; six millions in 1843; five millions in
        1844; four millions in 1845. It is further stipulated, that Interest, at
        the rate of five per cent per annum, shall be paid by the Government of
        China on any portions of the above sums that are not punctually
        discharged at the periods fixed.
	\begin{flushright}
		\autocite{Mayers:1902}
	\end{flushright}
\end{mdframed}
 % TREATY OF NANJING, TeX-EMBED

The treaty, signed \nth{29} August 1842 to come into effect on \nth{26} June
1843 \autocite{Saw:1975}, possesses unparalleled potential usage when conducting
an investigation into the causes of the downfall of the Qing empire. Despite its
the fact that its implementation precedes the events of 1912 by seventy years,
the effects of these articles represent the first significant international
tension which China endured, causing some scholars to argue that the treaty was
the initiator and catalyst for the downfall. The three selected articles
exasperate the true strain which was placed on the empire; not only losing
control of Hong Kong to the British crown, but also the forfeiture of
\$21,000,000 to \textit{Her Majesty}\footnote{The penalty of \$21 million to
\textit{Her Majesty} was broken down as follows: \$12 million for military
expenses, \$6 million for the opium which China had destroyed, and \$3 million
for the repayment of Chinese merchants' debts to the British
\autocite{Hsu:1999}.} in addition to the free movement of British merchants
throughout multiple Chinese ports on the mainland. As Britain suffered no
obligations in return, the treaty became known as the first of the
\textit{Unequal Treaties} signed between the Qing and Britain
\autocite{Hoe:1999}. Some scholars also state that as ``Missionaries could
preach within the empire [\ldots], literary undertakings rather gave way to
chapels, schools, and hospitals'' \autocite{Britton:1933}, implicitly suggesting
that the opening of Hong Kong and the five ports allowed the empire to
prosper\footnote{An exemplar of the ways in which Hong Kong prospered is the
fact that during the period 1841 to 60, ``the number of English and Chinese
periodicals and newspapers published in Hong Kong was more than the total number
in the rest of China'' \autocite{Huang:2001}.} under British rule, hence
undermining the authority and respect which had previously been owned solely by
the Qing emperor.

According to \autocite{Hsu:1999}, ``[the] treaty was imposed by the victor upon
the vanished at gunpoint, without the careful deliberation usually accompanying
international agreements in Europe and America''.  This comment on provenance
increases the value of the source beyond its verbatim content, as the
circumstances in which it was signed, where China were referred to as the
``vanished [nation]'', amplify its use into the realms of rendering the dynasty
as a completely powerless and helpless state at the whim of the British.

While the presented document is an undoubtedly vital piece of evidence for an
analysis into the factors which led to the Qing downfall, it suffers from a
similar limitation effecting all treaties and legal documents. This focuses on
the fact that the agreements outlined in the document may not be non-distinct
from the interpretation by the public of each empire, hence effecting the
societal morale towards the dynasty.  Due to the vast differences in the Chinese
and English languages, it is clear that the Chinese translation, a product of
the Qing society, was presented to the emperor in a heavily re-worded fashion,
favouring China. While it maintained the core points of the agreements, the text
suggests that the emperor is graciously providing the West with the
aforementioned privileges and maintains absolutely no implication of a forceful
signing. Hence, while this invaluable document explains the reality of the legal
situation, it cannot be used as a descriptor for an analysis into the morale of
the Chinese population toward the Qing: a rather significant element in their
ultimate downfall.

An exemplar of this concerns the clause regarding the cession of Hong Kong. In
the Chinese text, it is euphemistically stated that ``the emperor graciously
grants a place of rest and storage to the British after their long voyage to
China'' \autocite{Zhang:2007}. Similar language is used throughout the treaty.

Finally, the following newspaper column, written by the \textit{Special
Correspondent} to China for \textit{The Times} and published in August, 1884,
provides a deep insight into the situation of China from the perspective of a
foreigner. The column outlines the seldom amount of hope that is left for the
reduction and eventual abolition of opium smoking within China, stating that
Parliamentary-intervention would be meaningless, and any changes must be invoked
by the Chinese population-at-large.

\begin{fancyquote}
	[\ldots] The Chinaman, like men of other races, insists upon indulging in some stimulant or narcotic, and he has chosen opium. He is by no means the teetotaller which he is credited to be. Temperance societies exist in China. Still the Chinaman generally does not indulge in beer or wine --- a great debarrent being the cost when delivered from Europe --- and his \textit{samshu}\footnote{\textit{samshu}, mainly known as \textit{baiju} to natives (\jap{白酒} in Chinese), is a traditional Chinese spirit that is extremely strong at approximately 90 percent proof
	\autocite{Antkiewicz:1993}.} is a weakly subterfuge. The vice which it pleases him to indulge in is, therefore, opium. We have not yet succeeded in introducing temperance, far less abstinence, into England. And you may as soon expect the average Briton to give up his beer or spirits as the Chinaman his pipe. In neither case can you make man moral by Act of Parliament. No reform, I feel certain, is likely to come from the mandarinate, who are nearly without exception slaves to the habit, while the few free from it are powerless against it. It must come, if ever it does, by social reform, from the people themselves. The import of Indian opium by our Government is said by the missionaries to create a source of considerable difficulty in their relations with the Chinese. If not altogether a sincere belief with Chinamen, it is at least a highly convenient argument, and much used by them, that we are largely responsible for the prevalence of the habit; and not only the officials and \textit{literati} [the far-Eastern intelligentsia], but not a few foreigners even, have done their best to foster the idea. True or not true, the charge is one difficult to meet so long as Government preserves its present attitude with regard to Indian opium. Having in view the facts brought forward in this letter --- though of opinion that the suppression of the Indian opium traffic will not stop nor even diminish its use --- I think Government should take steps to discontinue it, and replace it by some other means of revenue. It can hardly be called a creditable source of revenue. From the practical financial as apart from the moral or sentimental aspect, it is advisable to examine the question and seek some means of replacing it. If no such steps be taken, Government will have lost the opportunity of carrying a measure of progress --- an act of self-respect as well as expediency --- for the import of Indian opium into China is doomed. [\ldots]
	\begin{flushright}
		\autocite{SpecialCorrespondent:1884}
	\end{flushright}
\end{fancyquote}
 % NEWSPAPER COLUMN, TeX-EMBED

In the column, the writer illustrates the extent to which opium addiction is
widespread throughout China by making a comparison to the entire British
population achieving abstinence to beer and wine. This renders this source
extremely useful due not only to its unsettling content, but also due to the
high and established status of the writer.

Written over twenty years since the conclusion of the \textit{Second Opium
War}\footnotemark, this article explains how morality cannot be achieved by mere
legislation by either nation, but instead must come from the desires of the
population. While much of the column is spent discussing the trade from a
financial and economics point-of-view, such that the writer explains why the
revenue generated from the trade is incredibly unreliable and unsustainable, the
points raised regarding the social implications greatly compliment the
usefulness of the document when analysing the reasons of the Qing dynasty's
demise.

\footnotetext{The \textit{Second Opium War} was fought just over four years from
\nth{8} October 1856, to \nth{24} October 1860 and had a subject matter rather
unrelated to the opium trade. It was initially caused by Chinese officials
boarding the Hong Kong/British ship, the \textit{Arrow}, and removing twelve
Chinese men \autocite{Wong:2002}; this is now known as the \textit{Arrow
incident} \autocite{Wong:1974}.  The Canton officials later claimed that it was
due to the \textit{Arrow} ship being accused of piracy, however in actuality,
they simply tore down the British flag and forced the vessel to effectively
surrender \autocite{Feige:2008}. Eventually resolved by the \textit{Treaty of
Tientsin}, in which concessions were granted to the Western nations of Russia,
France, the United Kingdom, and the United States \autocite{Nield:2015}, the
so-called ``unequal treaty'' has been described as a continuation to the
\textit{Treaty of Nanjing} \autocite{Wang:2008}. In particular, Britain were
granted rights to station representatives in Beijing \autocite{HKPress:1912},
even gaining access to the \textit{Forbidden City} \autocite{Dong:2004}. In
addition to the revocation of all anti-religious Qing laws
\autocite{Chassiron:1861}, Russia were also granted the rights to use maritime
trading ports \autocite{Adamov:1952}.

Knowledge of the \textit{Treaty of Tientsin} is vital when analysing post-1860
documents, as the ``China proper'' was no longer under the effective control of
the Qing, but instead the various aforementioned Western nations.}

Whilst the status and identity of the \textit{Special
Correspondent}\footnote{The true identity of the \textit{Special Correspondent}
has been investigated via personal correspondence with \textit{The Times}
reference library, however archival records have proven to not provide a
definite identification of the writer. It is rather safe to assume, by
extrapolating from certain contemporary documents, that the writer is
\textit{Sir James George Scott}. His obituary published on \nth{6} April, 1935
states that he went to ``Tongking as  [the] special correspondent for
\textit{The Times} in 1884 to describe the French operations there''
\autocites{TheTimes:1935}{Scott:1885}.} is undoubtedly instrumental when
analysing the accuracy and intentions of the document, the content of the column
reveals a rather anti-Anglo stance seen by the description, tone, and
personification of opium as something to be ``free from'' and ``powerless
against''. This anti-patriotic opinion makes the document increasingly useful
due to the additional effort required of the writer to adopt, and publish, such
a critical view of an official British activity.

Despite not being immediately obvious, the content of the column proves
invaluable for the analysis, as it describes an ``enslaved'' general population:
a perfect device for a weak and rapidly declining leadership
\autocite{Rotberg:2003}. The \textit{Special Correspondent} also cites the
missionaries' opinions regarding the extent to which the trade severs any
potential relations between Britain and the Qing\footnote{While the
\textit{Special Correspondent} accepts this view may be fabricated, he also
accepts that it is a convenient excuse often used by Chinese diplomats who are
spiteful of the entire concept of the opium trade.}, with some scholars citing
that despite the vast differences in the cultural aspects of the two empires, a
positive relationship is necessary for bilateral continued success succeeding
the \textit{Unequal Treaties} \autocite{Fairbank:1942}. The combination of these
factors quickly leads to the conclusion that the already-weakened Qing (as a
result of the opium wars) are promptly disintegrating as a result of the
declining population-morale in addition to the severing of potentially positive
British-Qing relations at a diplomatic level.

Similar to a treaty document, however, any newspaper column suffers the
limitations of being a product of the society in which it was produced: Britain:
the arguably-primary aggressor. Despite the anti-Anglo and anti-opium stance
communicated in the column, it is clear that the \textit{Special Correspondent}
largely places the blame of widespread addiction more on the Chinaman himself,
rather than the traders acting on behalf of the \textit{East India Company},
claiming they have ``chosen opium''. An empathetic and more realistic viewpoint
would be that it is the fault of the Western powers who introduced an unknown
substance into a \textit{de-jure} Confucianist \autocite{Billioud:2007} nation.
While this does not directly effect the reality of the document when analysing
the reasons for the Qing downfall, the limitation should certainly be considered
when using this document as a descriptor for the general morale within China.

Historians are generally divisive on the primary cause of the Qing downfall,
however it is a uniformly accepted hypothesis that there were a vast array of
causes. A common argument, conforming to the ideas put forward in the
introduction, state that the opium wars and the subsequent foreign intervention
were the primary causes. While there are many oppositions to this particular
hypothesis, there is widespread acceptance that the opium wars, while perhaps
not the primary factor, undoubtedly took the role as a significant actor in the
dynastic demise.

A common argument among scholars focuses on the socio-economic consequences that
ensued as a result of the widespread consumption and purchase of opium
throughout the Chinese empire. In his 2000 recount of the complex network of
trading events events throughout China, \textit{Motono} presents a
somewhat-revisionist argument, proclaiming that ``English-speaking Chinese'',
created as a product of the cession of Hong Kong, which was subsequently due to
the \textit{Treaty of Nanjing} for the \textit{First Opium War}, were central in
the introduction of Western customs into China.

Redefining ``Western intervention'', \textit{Motono} presents a view with the
basis that the nature of the Western interference altered drastically succeeding
the 1880s, arguing that it was no longer founded on ``gunboat diplomacy'' or the
importation of foreign goods, but rather the Westernised Chinese. With this
argument, \textit{Motono} implicitly states that after the beginning of Western
influence, especially relating to the cession of Hong Kong, the Western impacts
begun to arise largely from the Chinese: many of whom had become accustomed to
the English language and culture\footnote{It could further be extrapolated that
due to the increase of Westernised Chinese, the Western influence become almost
entirely autonomous, and that the literal intervention from Britain acted as an
undoubtedly critical but merely initial catalyst for the bourgeois-based
revolution.}.

\begin{mdframed}[style=fancyquote]
        [\ldots] The Western impact on China from the 1880s onwards was entirely
        different from that which preceded it. It did not consist of gunboat
        diplomacy of imported foreign goods, such as Indian opium or British
        machine-made cotton goods, let alone advanced Western civilisation.
        Previous studies, which supposed that `Western impact' was the
        above-mentioned were based on a false assumption that only Westeners
        could produce a Western impact on China.

        In fact, the Chinese themselves were capable of promoting a Western
        impact on China. The crucial change in treaty-port society in China
        during the 1880s was the emerge of a new type of Chinese people who
        could speak and write English fluently and were familiar Western
        culture. According to [an] article in the \textit{North-China Herald},
        Western residents in China did not entirely trust and welcome these
        `English-speaking Chinese', due to their ambiguous
        character\footnote{The term ``ambiguous character'' can be used to
        intensify the extents to which both the British and Chinese felt their
        respective cultures were wholly incompatible, and that the absurd
        proposition of an ``English-speaking Chinese'' was not true to either
        culture. \autocite{Lyczak:1976}}. [\ldots]

        So, according to my new definition, `the Western impact on China' after
        the 1880s consisted of the arrival of Western institutions stipulated by
        the `unequal' treaties and the growth of commercial networks of
        English-speaking Chinese who cooperated with foreign firms in order to
        benefit from the treaty privileges. With these privileges, the
        English-speaking Chinese and foreign firms could develop their
        commercial networks. [\ldots]

        Why could the commercial activities of the English-speaking Chinese
        undermine the apparently unshakable solidarity of the prominent Chinese
        merchants' groups and thus threaten the control the Qing government
        officials exercised over them, especially from the latter half of the
        1890s onwards? [\ldots]
	\begin{flushright}
		\autocite{Motono:2000}
	\end{flushright}
\end{mdframed}

 % MOTONO ARGUMENT, TeX-EMBED

\textit{Eiichi Motono}, native name \jap{本野英一}, is a Japanese professor in
the \textit{School of Political Science and Economics} at the prestigious
\textit{Waseda University} in Tokyo. His 2000 revisionist book was published in
both Britain by \textit{MacMillan Press Ltd.}, and in the United States by
\textit{St.\ Martin's Press, Inc.} The date and publication locations of this
book are incredibly advantageous with respect to its credibility. While it was
published in the nations of the original aggressors in the opium conflict, it
would be foolish to suggest that such a recent publication suffers from any
overtly pro-British or pro-American opinions. However, it should simultaneously
be considered that the author is of a Japanese heritage, and therefore may have
feelings, whether subconscious or otherwise, of animosity towards China due to
the altercations suffered between the two nations during the series of the
\textit{Sino-Japanese Wars}\footnote{The \textit{First Sino-Japanese War} was
fought between China and Japan from \nth{25} July, 1894 to \nth{17} April, 1895.
As previously mentioned, some scholars, such as \textit{Paine}, suggest that the
crushing defeat of China led to Japan become the dominant force in East Asia for
the first time in modern history. Primarily fought over control of Korea
\autocite{Olender:2014}, which had previously been owned by China, the war led
to the complete humiliation of the Qing forces in their own ``back garden''. The
\textit{Second Sino-Japanese War} was fought much later from \nth{7} July, 1937
to \nth{2} September, 1945. Although irrelevant for this analysis, it has been
described as the largest Asian war in the twentieth century \autocite{Bix:1992}.
Heterologous to the \textit{First Sino-Japanese War}, it concluded with a
Chinese victory and Japanese surrender \autocite{Kitamura:2014}.}, which, by
some accounts, has been stated to have been another major-contributing factor to
the 1911 revolution \autocite{Paine:2009}.

Despite the fact of his heritage and potential reservations, the prestige of
\textit{Motono} increases upon analysis of his previous publications while at
\textit{Waseda}. Writing in both Japanese and English, \textit{Motono} has
published over one-hundred works covering all aspects of modern Chinese history,
ranging from the effects of the \textit{American Civil War}
\autocite{Motono:1984} to in-depth legal analyses of the Sino-British `mixed
court' system \autocite{Motono:1996}\footnote{The Sino-British `mixed court'
which is being referred to here was a controversial issue between Britain and
the Qing during the \textit{First Opium War}, beside the obvious. Serving as an
exemplar for the diplomatic-political term of \textit{extraterritoriality}, the
British, dissatisfied with the legitimacy of the Qing legal system, decided to
place select Britons in the court of the Qing. The aforementioned mistrust of
the Qing legal system likely arises from the controversial 1784 case in which a
Briton working on behalf of the \textit{East India Company} was executed as a
result of accidentally killing a Chinese \autocite{Cassel:2012}.}.

In the light of his major contributions to various international journals in
addition to the authorship of several books covering a large variety of modern
Chinese history, \textit{Motono}'s presented thesis is incredibly well-balanced
and convincing due to his prestige as a modern Sinologist with absolute freedom
of speech and uncensored, international publishing rights.

This argument, however, by no means enjoys uniformity among modern
historians\footnote{A renowned \textit{Stephen D. Krasner} puts forward the idea
that the Qing dynasty did not only survive the two opium-based conflicts, but
also reassert themselves, using the example of foreign representatives been
``admitted into the presence of the emperor'' in 1874 \autocite{Krasner:2001}.}.
\textit{James T. K. Wu}, a member of the \textit{Far-Eastern Institute} in the
\textit{University of Washington}, published his thesis in 1948 to be read at
the annual meeting of the \textit{Pacific Coast Branch of the American
Historical Association} in Seattle, Washington. His paper is an excellent
argument based on the \textit{Taiping Rebellion} and its subsequent effects on
the Qing decentralisation.

Since the early \textit{Zhou} dynasty (c. 1046 century B.C.
\autocite{Loewe:1999}), the term \textit{Mandate of Heaven} has been used
amongst leaders to justify their power \autocite{Zhao:2009}, insisting that
their leadership and authority is supported by Heaven.  Throughout history it
has since been stated that natural disasters, such as floods and earthquakes,
are sent by Heaven to show a loss of faith in the current emperor, and
subsequently, the immediate requirement for a new leader \autocite{Elvin:1998}.
This is an incredibly important concept when studying the imperialist dynasties
of China, as Heaven's endorsement of a dynasty's legitimacy was often central in
both their rise to power and eventual demise.

The \textit{Taiping Heavenly Kingdom} was an unrecognised monarchy within China
from 1851 to 1864, largely run by a self-proclaimed brother of \textit{Jesus
Christ} \autocites{Reilly:2014}{Yap:1954}. Literally translating from its
Chinese name \zht{天國} as \textit{[the] Kingdom of Heaven}, its leader,
\textit{Hong Xiuquan}, sought to replace the traditional practices of
Confucianism and Buddhism with Christianity \autocite{Spence:1996}.
Understandably, this quickly led to an anti-Qing rebellion orchestrated by
\textit{Hong} and his supporters. Although the Qing enjoyed victory in 1864, the
scar left by what is now known as the \textit{Taiping Rebellion} is often used
by scholars as the driving force for presenting alternative hypotheses. Despite
its early origins as developing out of frustration for the Qing's handling of
the Western ``invasion'', the \textit{Taiping Rebellion} is generally considered
as a separate event which is almost entirely independent of the opium-based
conflicts.

\begin{fancyquote}
	The \textit{Taiping Rebellion}, which lasted from 1850 to 1864, represents the type of rebellion which possessed revolutionary characteristics. It aimed not only at unseating the Manchus from the throne, but sought to establish a new economic and social order. Although the \textit{Taiping Rebellion} was finally suppressed, its impact upon the existing Manchu order, especially its fiscal system, was great. The Manchu government was forced by the rebellion into a series of changes and reforms which fundamentally altered the power structure of China and paved the way for the revolutionary movements of the following century. [\ldots]
	\begin{flushright}
		\autocite{Wu:1950}
	\end{flushright}
\end{fancyquote}
 % WU ARGUMENT, TeX-EMBED

The paper is a product of \textit{Modern Chinese History Project} undertaken by
the \textit{Far-Eastern and Russian Institute of the University of Washington}.
Although of a Chinese heritage, Wu's argument is likely to be free of any
unjustified and overtly pro-China and anti-Western elements due to his
association with the prestigious \textit{Far-Eastern and Russian
Institute}\footnote{The aforementioned \textit{Far-Eastern and Russian
Institute} at the \textit{University of Washington} was headed by the prolific
\textit{George Edward Taylor}, an expert of the \textit{Taiping Rebellion} and
Chinese politics at the time of the Qing's dynastic decline
\autocite{Taylor:1999}.}.

While both on-line and off-line records of \textit{Wu} are incredibly limited,
his institution was an extraordinarily well-respected organisation during the
last century, being generally regarded as the authoritative source on modern
China. It is commonly regarded that, although the Qing managed to defeat the
rebellion, subsequently ending the \textit{Taiping Heavenly Kingdom}, they only
did so at a great cost: further increasing the decentralisation of provisional
authority figures, hence granting autonomy to a greater number of
``sub-states'', thus increasing the possibilities of a rebellion not only having
the ability to begin, but also to garner large followings without the knowledge
of the Qing. In \textit{Wu}'s largely-revisionist theory of the impacts of the
\textit{Taiping Rebellion} upon the Manchurian fiscal
system\footnote{\textit{Wu} also states in the introduction that due to the
simultaneously centralised and decentralised nature of the pre-rebellion
Government, China's administrative structure was very difficult to define
properly. The sudden decentralisation of the fiscal system allows for an easier
analysis.}, it is argued that the decentralisation of the enforcement agencies
of China's fiscal policy caused the centralised state authorities in
Peking/Beijing to lose power over all but the larger cities, in which they were
geographically neighbouring.

This thesis was published in 1948, shortly succeeding the \textit{Second World
War} (1939 to 1945) and on the brink of the \textit{Cold War} between Russia and
America \autocite{Leffler:2010}. With a history of Sino-Soviet relations
throughout the very recent years \autocite{Garver:1989}, the animosity towards
Russia in America often extended to China. Although not in its prime until the
1950s, \textit{McCarthyism} and \textit{The Red Scare} also contributed towards
a general anti-Sino stance with the American populous due to their involvement
in the Korean war \autocite{Schrecker:1998}.

Although the thesis is relatively unlikely to suffer from great extents of
exaggeration and deception in comparison to opinions from historians and authors
with a more significant level of impartiality, any document which is a product
of a society such as America during such a volatile time period must be analysed
with the appropriate reservations. It is also critical to recall that America
played a much greater role than an innocent bystander in the opium trade and
demise of the Qing, as they were one of the key nations who signed an
\textit{Unequal Treaty} with the Qing
\autocite{Peters:1961}\footnote{Additionally, it should be mentioned that in
article eleven of the American \textit{Treaty of Tientsin} with the Qing, it is
stated that all American citizens are granted the right to roam freely
throughout the entirety of China for both business and pleasure, providing they
possessed a valid passport.  They were also involved in the mass-shipment of
opium to China, second only to the British. According to many estimates, out of
the 4500 chests of opium shipped to China in 1817, 1900 were from America
\autocite{Hu:1991}.} succeeding the \textit{Second Opium War}
\autocite{Johnstone:1937}.

While both of these theses simultaneously enjoy prestige and suffer from
limitations, it is often the case that historians' arguments and accounts of
events often possess usage based upon their positive recency: the inverse of
contemporary sources. This phenomenon can mainly be attributed to the easing of
tensions as time progresses along with the increasing availability of documents
on which to base research. An exemplar of this can be seen by the fact that it
is extremely unlikely that a 1948 writer in America had the absolute freedom to
condemn the actions of the West against a current mutual enemy. Conversely, a
2000 writer in Japan likely had the means and freedoms to explain and define the
situation as he saw fit, basing his analysis on widely available documents.

Due to the anti-China morale being adopted in America as a result of the
anti-Soviet stance, in addition to the history of \textit{Unequal Treaties}
between America and the Qing, reasonable amounts of scepticism must be cast upon
documents published in times of political volatility.  While both authors have
published specialised works in modern Chinese history, \textit{Motono}'s
significant contribution to the field in addition to his recent thesis renders
him of a prestige which \textit{Wu} and the 1948 \textit{University of
Washington} cannot match in a modern analysis of such controversial events.
Because of this, the initial argument outlined in the introduction is
maintained, such that while it is not suggested nor implied that the opium wars
were the sole or `final' factor leading to the Qing's dynastic decline, the
extraordinary amount of evidence, both from historians and contemporary
documents communicating the magnitude of the conflicts, allows a logical
conclusion to be reached that they were the clear primary factor.

\vfill
\noindent\ignorespacesafterend%
\textbf{%
        Word Count, including abstract: \wordcount\\
        Word Count, excluding abstract: \wordcountabstract\\[1em]
}
Verbatim sources, footnotes, and the bibliography have been omitted from the
word count.

\pagebreak
\printbibliography[title={Cited Works}, heading=bibintoc]

\end{document}

