% The Opium Wars, Oliver Dixon
% History Coursework Dissertation LaTeX source (designed for PDFLaTeX)

\documentclass{article}

\usepackage[a4paper, total={6in, 10in}]{geometry}
\usepackage[bottom]{footmisc} % make footnotes look like footnotes
\usepackage{pinyin,multicol,graphicx,nth}

\usepackage[
	backend=biber,
	sorting=none,
	bibencoding=utf8,
	style=alphabetic
]{biblatex}

\addbibresource{main.bib}
\defbibheading{bibliography}{\section*{Cited Works}}

\usepackage{hyperref}
\hypersetup {
	colorlinks = true,
	allcolors = {blue}
}

\setlength{\footnotesep}{1em}

% thin horizontal line spanning the width of text
\newcommand{\textrule}{\noindent\makebox[\linewidth]{\rule{\linewidth}{0.4pt}}}

% fancyquote: \fancyquote{<quote text>}{<author(s)>}{<date>}{<citation>}
\newcommand{\fancyquote}[4]{
	\begin{quote}
		\textrule #1
		\begin{flushright}
			\textit{#2 (#3)} #4 \textrule
		\end{flushright}
	\end{quote}
}

\title{``\emph{The Opium Wars were the Primary Reason for the Downfall of the Qing Dynasty in the Years 1814-1912}'' --- A Historical Analysis}
\author{Oliver Dixon}
\date{Summer 2019}

\begin{document}

\maketitle

\begin{abstract}

	Succeeding the Ming dynasty, the great Qing dynasty ruled from 1644 to 1912, achieving the most significant strides ever witnessed on the Chinese political spectrum. Aside from doubling the land under the empire
	\autocite{Turchin:2006} while tripling the population through encapsulation of a broad range\footnote{Such ethnicities include the Uyghur Muslims native to Xinjiang, Tibetans, Mongols, and Burmese.} of ethnicities within the Chinese state
	\autocite{Rowe:2012}, the dynasty enjoyed economic superiority of proportions that Europe, the Americas, and their other Asian neighbours could only imagine 
	\autocite{Maddison:2007}. Despite their initial prosperity and success as an irreplaceable and integral element of the rapidly-increasing worldwide economy, the Qing empire suffered the fate of taking the place as the final imperial dynasty of China due to an array of factors, both international and domestic.

	This essay will discuss the factors which led to the downfall of the greatest dynasty in the history of China. Also under examination will be the extent to which the strained international trade relations with the East India Company and the ensuing Opium wars contributed to the ultimate demise of 1912.

\end{abstract}

\textrule
\vspace*{0.6em}
{\centering \textsc{This ``Non-Examined Assessment'' Document is Presented as Partial Fulfilment of the \textit{A.Q.A. A-Level History} Qualification}\\}
\textrule
\vspace*{0.6em}

\begin{multicols}{2}

\section{Introduction}

From the exposure of the weaknesses in the Qing military to the anti-imperialist and anti-feudalism domestic rebellions, it is clear that the Opium wars only ever existed as a subset of the immediate issues threatening the Great dynasty. The ever-diminishing national identity in the face of foreign intervention is undoubtedly a significant factor when inspecting the ultimate demise of China's final dynasty, however the somewhat-inadvertent internationalisation efforts of the global powerhouse cannot be solely attributed to the onset of the Opium wars.

Since their ascension to power in 1644, the Qing dynasty has incorporated various non-Manchurian populations into the China proper, leading to an inevitably-factionalised society. Extending to the military of the Qing, the 240,000-strong Japanese armies brutally humiliated the 600,000-strong Chinese forces during the first Sino-Japanese war
\autocite{Fenby:2013}, exposing the widespread incompetence and lack of training possessed by the Sino armed forces
\autocite{Jowett:2013}.

However, the Opium wars cannot be ignored as a major contributing factor. The first Opium war initiated a state of mass-unrest throughout the region, invoking what some have called a ``\textit{Chinese people's bourgeois-democratic revolution against imperialism and feudalism}''
\autocite{Janin:1999}, which is incredibly threatening due to its direct contradiction with the principles and philosophies of the imperialist Qing.

Some, however, have argued that the importation of opium was not necessarily the primary cause of the fall of the empire, but rather the domestic Boxer rebellion and subsequent instantiation of the Eight-Nation alliance\footnote{The nations involved in the anti-Sino alliance were Japan, Britain, France, the United States, Russia, Germany, Italy, and Austria-Hungary. They were also supported by Australian and Indian forces of the British empire
\autocite{Gardener:2016} --- this will be covered in more detail throughout the essay.} which has been described as ``bringing the Great dynasty to its knees''
\autocite{Mitchell:2008}.

\end{multicols}

\pagebreak
\printbibliography

\end{document}
