% "The Opium Wars were the Primary Reason for the Downfall of the Qing Dynasty in the Years 1814-1912"
% Oliver Dixon, History Coursework Dissertation LaTeX source (designed for PDFLaTeX)

\documentclass{article}

\usepackage[a4paper, total={6in, 10in}]{geometry}
\usepackage[bottom]{footmisc} % make footnotes look like footnotes
\usepackage[super]{nth}
\usepackage{graphicx,mdframed,xcolor,indentfirst,fancyhdr,titling,CJKutf8}

% \usepackage[firstpage]{draftwatermark}
% \SetWatermarkScale{2}
% \SetWatermarkLightness{0.8}

% --- BIBLIOGRAPHY SET-UP --- %

\usepackage[
	backend=biber,
	sorting=none,
	bibencoding=utf8,
	style=alphabetic
]{biblatex}

\addbibresource{main.bib}

% Break long U.R.L.s which do not contain breaking characters
% -- from https://tex.stackexchange.com/a/134281
\setcounter{biburllcpenalty}{7000}
\setcounter{biburlucpenalty}{8000}

\newcommand{\jap}[1]{\begin{CJK}{UTF8}{min}\normalfont #1\end{CJK}}

% --- HYPERREF IMPORT AND SET-UP --- %

\usepackage{hyperref}
\hypersetup {
	colorlinks = true,
	allcolors = {blue}
}

% --- FANCYHDR SET-UP --- %

\fancypagestyle{stdhdr}{
	\fancyhf{}
	\fancyhead[R]{\textsc{\theauthor}}
	\fancyhead[L]{\textit{A.Q.A. A-Level History} Non-Examined Assessment}
	\fancyfoot[R]{Page \thepage}
}

\fancypagestyle{titlehdr}{
	\fancyhead{}
	\renewcommand{\headrulewidth}{0pt}
}

% --- ENVIRONMENT AND COMMAND SET-UP --- %

% thin horizontal line spanning the width of text
\newcommand{\textrule}{\noindent\makebox[\linewidth]{\rule{\linewidth}{0.4pt}}}

\newcommand{\articlehead}[1]{
	\vspace*{0.5em}
	\begin{center}
		\textsc{#1}
	\end{center}
	\vspace*{0.5em}
}

\newenvironment{fancyquote}
	{
		\begin{mdframed}[linecolor=lightgray]
	}
	{
		\end{mdframed}
	}
	
% \tolerance=10000 % "infinite" tolerance (see TeXBook page 29)
\emergencystretch=1em
\setlength{\footnotesep}{1em}

\title{``\emph{The Opium Wars were the Primary Reason for the Downfall of the Qing Dynasty in the Years 1814-1912}'' --- A Historical Analysis}
\author{Oliver Dixon}
\date{Summer 2019}

% ---------------------- %
% --- DOCUMENT START --- %
% ---------------------- %

\begin{document}

\clearpage\maketitle
\thispagestyle{titlehdr}
\pagestyle{stdhdr}

\vspace*{-1.5em}
\begin{figure}[h!]
	\centering
	\def\svgwidth{0.5\linewidth}
	\input{chinese_frontcover.pdf_tex}
\end{figure}
\vspace*{-0.5em}

\begin{abstract}

	Succeeding the Ming dynasty, the great Qing dynasty ruled from 1644 to 1912, achieving the most significant strides ever witnessed on the Chinese political spectrum. Aside from doubling the land under the empire
	\autocite{Turchin:2006} while tripling the population through encapsulation of a broad range\footnote{Such ethnicities include the Uyghur Muslims native to Xinjiang, Tibetans, Mongols, and Burmese.} of ethnicities within the Chinese state
	\autocite{Rowe:2012}, the dynasty enjoyed economic superiority of proportions that Europe, the Americas, and their other Asian neighbours could only imagine 
	\autocite{Maddison:2007}. Despite their initial prosperity and success as an irreplaceable and integral element of the rapidly-increasing worldwide economy, the Qing empire suffered the fate of taking the place as the final imperial dynasty of China due to an array of factors, both international and domestic.

	This essay will discuss the factors which led to the downfall of the greatest dynasty in the history of China. Also under examination will be the extent to which the strained international trade relations with the East India Company and the ensuing Opium wars contributed to the ultimate demise of 1912.

\end{abstract}

\textrule
\vspace*{0.6em}
{\centering \textsc{This ``Non-Examined Assessment'' Document is Presented as Partial Fulfilment of the \textit{A.Q.A. A-Level History} Qualification}\\}
\textrule

\section{Introduction}

From the exposure of the weaknesses in the Qing military to the anti-imperialist and anti-feudalism domestic rebellions, it is clear that the Opium wars only ever existed as a subset of the immediate issues threatening the Great dynasty. The ever-diminishing national identity\footnote{According to contemporary British travellers to China, a true Chinaman often despised all other nations than his own, while considering foreigners to be ``barbarians''
\autocite{McPherson:1842}. These notes were published in the midst of the \textit{First Opium War} however, hence the general animosity towards foreigners was likely to have been understandably high during the traveller's expedition.} in the face of foreign intervention is undoubtedly a significant factor when inspecting the ultimate demise of China's final dynasty, however the somewhat-inadvertent internationalisation efforts of the global powerhouse cannot be solely attributed to the onset of the Opium wars.

Despite not existing as a sole catalyst for the Qing downfall, the Opium wars bore such significance\footnote{Historians W. Travis Hanes III, Ph.D. and Frank Sanello communicate the magnitude of the Opium wars by comparing the nineteenth-century events to the modern-day U.S. government legalising cocaine and allowing its uncontrolled trade succeeding a defeat in a military offensive launched by the cocaine cartels of Columbia
\autocite{Hanes:2004}.} in the making of modern China that it would be improper to suggest they were anything other than the primary causes.

The first Opium war initiated a state of mass-unrest throughout the region, invoking what some have called a ``\textit{Chinese people's bourgeois-democratic revolution against imperialism and feudalism}''
\autocite{Janin:1999}, which is incredibly threatening due to its direct contradiction with the principles and philosophies of the imperialist Qing. Various wars fought between the Qing empire and Western powers ultimately led to what are now referred to as the ``Unequal Treaties'', in which the Qing was diplomatically forced to sign a series of unfavourable agreements with the Western powers \autocite{Wang:2005}.

Additionally, some have argued that the importation of opium was not directly attributable for the downfall of the empire, but rather the Boxer rebellion and subsequent instantiation of the Eight-Nation alliance, which is an indirect consequence. Historians whom argue this perspective largely cite the ``Boxer Protocol''. Often considered to be included in the series of Unequal Treaties imposed upon the Qing, \footnote{The nations involved in the anti-Qing alliance were Japan, Britain, France, the United States, Russia, Germany, Italy, and Austria-Hungary. They were also supported by Australian and Indian forces of the British empire
\autocite{Gardener:2016}.} the protocol was the driving force which ultimately ``[brought] the Great dynasty to its knees'' due to crippling debt \autocite{Mitchell:2008}.

However, it would also be improper to avoid considering the events which were not directly attributable to the West, shifting the focus to China's eastern neighbours. Since their ascension to power in 1644, the Qing dynasty has attempted to incorporate various non-Manchurian populations into the China proper, leading to an inevitably-factionalised society. Extending to the military of the Qing, the 240,000-strong Japanese armies brutally humiliated the 600,000-strong Chinese forces during the first Sino-Japanese war
\autocite{Fenby:2013}, exposing the widespread incompetence and lack of training possessed by the Sino armed forces
\autocite{Jowett:2013} when fighting for influence in the Korean peninsula.

\section{Analysis of Primary Sources}

\subsection{A Letter of Advice from Lin Tse-Hs\"u to Queen Victoria}

Lin Tse-Hs\"u, also known as Lin Zexu, was the Chinese commissioner in Canton, sent by the Qing emperor\footnote{The Qing emperor, Daoguang, allegedly lost his son, Xianfeng, to an opium-induced overdose in 1861, possibly fuelling Daoguang's absolute hatred of the opium trade
\autocite{Ringmar:2013}. Xianfeng's death was incredibly controversial with anti- and pro-opium activists alike, submitting that the hypocrisy of his acts discredit all anti-opium opinions of the Qing. Before his death, he presented with \textit{massive haemoptysis} (the ``coughing-up'' of blood in excess of 600 millilitres
\autocite{Sabatine:2013}).} to negotiate an end to the importation of opium by the East India Company. In 1839, preceding the \textit{First Opium War} by three years, Lin wrote a strongly-worded letter of advice to Queen Victoria, stating his strong desire to end the importation of opium by any means necessary. The following is a short extract from the English translation of the letter.

\begin{fancyquote}
	\emph{[\ldots]} But after a long period of commercial intercourse, there appear among the crowh of barbarians both good persons and bad, unevenly. Consequently there are those who smuggle opium to seduce the Chinese people and so cause the spread of the poison to all provinces. Such persons who only care to profit themselves, and disregard their harm to others, are not tolerated by the laws of heaven and are unanimoly hated by human beings. His Majesty the Emperor, upon hearing of this, is in a towering rage. He has especially sent me, his commissioner, to come to Kwangtung, and together with the governor-general and governor jointly to investigate and settle this matter.\par
	All those people in China who sell opium or smoke opium should receive the death penalty. We trace the crime of those barbarians who through the years have been selling opium, then the deep harm they have wrought and the great profit they have usurped should fundamentally justify their execution according to law. We take into to consideration, however, the fact that the various barbarians have still known how to repent their crimes and return to their allegiance to us by taking the 20,183 chests of opium from their storeships and petitioning us, through their consular officer [superintendent of trade], Elliot, to receive it. It has been entirely destroyed and this has been faithfully reported to the Throne in several memorials by this comissioner and his colleagues. \emph{[\ldots]}
	\begin{flushright}
		\emph{\autocite{Teng:1979}}
	\end{flushright}
\end{fancyquote}
 % LETTER TO QUEEN VICTORIA, TeX-EMBED

This source is an extremely useful piece of contemporary evidence when conducting an investigation into any aspect of the Qing dynasty, as it is written with an extraordinarily strong sense of foreshadowing for a violent conflict\footnote{Some scholars have stated that this letter is a somewhat unique piece of attempted diplomacy from China, as the incredibly direct letter violates their usual conventions of writing in a ``highly-stylised'' fashion
\autocite{Kishlansky:1995}.} between the anti-opium groups and the \textit{East India Company}. It becomes increasingly-valuable when considering the author, recipient, and the circumstances in which the letter was written. Sent to Queen Victoria by the commissioner of the primary trading port of the \textit{East India Company}, Lin was likely in a more hopeful position to make a plea to \textit{Her Majesty} than any other entity in China due to his familiarity and knowledge on the matter. His extreme stance on serving the death penalty to distributors and users of opium represents the severity of the situation throughout China, while also exasperating the level to which opium-smoking was widespread to ``all provinces''.

The unfathomably-widespread nature of opium throughout all Chinese provinces renders Lin's strongly-worded advice as an inescapable point-of-interest, as it undisputedly provides evidence to show the fundamental scar that opium had left on the Chinese population, and the desperation of the Qing to restore order by any means necessary.

This source could be somewhat-limited due to the fact it was written by a high-ranking member of the Qing dynasty, whom had a strong inclination to exaggerate the effects of British presence in China in order to appease the Emperor, however the majority of his writings are known to be factual to a rather striking degree.

\subsection{The Treaty of Nanjing (Treaty of Peace, Friendship, and Commerce Between Her Majesty the Queen of Great Britain and Ireland and the Emperor of China)}

Often hailed as a ``turning-point in far-Eastern history''
\autocite{Fairbank:1940}, the \textit{Treaty of Nanjing}, also known as the \textit{Treaty of Nanking}\footnote{Due to the vast differences of the Chinese and English languages, various \textit{romanisation} methods were devised. ``\textit{Nanking}'' is the \textit{Wade-Giles} romanisation, whereas ``\textit{Nanjing}'' is the increasingly-popular \textit{Pinyin} romanisation. See \url{https://web.archive.org/web/20140714185035/http://www.sacu.org/roman.html} [archived] for more information.}, was a treaty between Britain and the Qing to end the \textit{First Opium War}. The general consensus regarding the treaty states that despite its undebatable harshness and inequality, the Qing representative (Qingyi) had no choice but to sign the agreement due to the their status as a ``defeated nation''
\autocite{Mao:2018}. The following extract is from the thirteen-article treaty, highlighting the most significant elements.

\begin{mdframed}[style=fancyquote]
	\articlehead{[\ldots] Article II}
        \textit{His Majesty the Emperor of China} agrees, that British Subjects,
        with their families and establishments, shall be allowed to reside, for
        the purpose of carrying on their mercantile pursuits, without
        molestation or restraint, at the cities and towns of Canton, Amoy,
        Foochowfoo, Ningpo, and Shanghai; and \textit{Her Majesty the Queen of
        Great Britain}, will appoint Superintendents, or Consular Officers, to
        reside at each of the above named Cities, or Towns, to be the medium of
        communication between the Chinese Authorities, and the said merchants,
        and to see that the just Duties and other Dues of the Chinese
        Government, as hereafter provided for, are duly discharged by
        \textit{Her Britannic Majesty}'s subjects.

        \vspace*{1em}
        \articlehead{Article III [\ldots]}
        It being obviously necessary and desirable, that British Subjects should
        have some Port whereat they may careen and refit their Ships, when
        required, and Keep Stores for that purpose, \textit{His Majesty the
        Emperor of China} cedes to \textit{Her Majesty the Queen of Great
        Britain}, the Island of Hong-Kong, to be possessed in perpetuity by
        \textit{Her Britannic Majesty}, Her Heirs and Successors, and to be
        governed by such Laws and Regulations as \textit{Her Majesty the Queen
        of Great Britain}, shall see fit to direct.

        \pagebreak % likely temporary
        \articlehead{[\ldots] Article VII [\ldots]}
        It is agreed that the Total amount of Twenty One Millions of Dollars,
        described in the three preceding Articles, shall be paid as follows: six
        millions immediately [\ldots]; six millions in 1843; five millions in
        1844; four millions in 1845. It is further stipulated, that Interest, at
        the rate of five per cent per annum, shall be paid by the Government of
        China on any portions of the above sums that are not punctually
        discharged at the periods fixed.
	\begin{flushright}
		\autocite{Mayers:1902}
	\end{flushright}
\end{mdframed}
 % TREATY OF NANJING, TeX-EMBED

The treaty, signed \nth{29} August 1842 to come into effect on \nth{26} June 1843
\autocite{Saw:1975}, possesses unparalleled potential usage when conducting an investigation into the causes of the downfall of the Qing empire. Despite its the fact that its implementation precedes the events of 1912 by seventy years, the effects of these articles represent the first significant international tension which China endured, causing some scholars to argue that the treaty was the initiator and catalyst for the downfall. The three selected articles exasperate the true strain which was placed on the empire; not only losing control of Hong Kong to the British crown\footnote{Somewhat entertainingly, the clause in the Chinese text euphemistically states that ``the emperor graciously grants a place of rest and storage to the British after their long voyage to China''
\autocite{Zhang:2007}. Such euphemisms and a general Qing-favourable writing style are commonplace in many of the \textit{Unequal Treaties} signed during the opium war-era.}, but also the forfeiture of \$21,000,000 to \textit{Her Majesty}\footnote{The penalty of \$21 million to \textit{Her Majesty} was broken down as follows: \$12 million for military expenses, \$6 million for the opium which China had destroyed, and \$3 million for the repayment of Chinese merchants' debts to the British
\autocite{Hsu:1999}.} in addition to the free movement of British merchants throughout multiple Chinese ports on the mainland. As Britain suffered no obligations in return, the treaty became known as the first of the \textit{Unequal Treaties} signed between the Qing and Britain
\autocite{Hoe:1999}. Some scholars also state that as ``Missionaries could preach within the empire [...], literary undertakings rather gave way to chapels, schools, and hospitals''
\autocite{Britton:1933}, implicitly suggesting that the opening of Hong Kong and the five ports allowed the empire to prosper\footnote{An exemplar of the ways in which Hong Kong prospered is the fact that during the period 1841-60, ``the number of English and Chinese periodicals and newspapers published in Hong Kong was more than the total number in the rest of China''
\autocite{Huang:2001}.} under British rule, hence undermining the authority and respect which had previously been owned solely by the Qing emperor.

According to
\autocite{Hsu:1999}, ``[the] treaty was imposed by the victor upon the vanished at gunpoint, without the careful deliberation usually accompanying international agreements in Europe and America''. This comment on provenance increases the value of the source beyond its verbatim content, as the circumstances in which it was signed, where China were referred to as the ``vanished [nation]'', amplify its use into the realms of rendering the once Great dynasty as a completely-powerless and helpless state at the whim of the British.

While the presented document is an undoubtedly vital piece of evidence for an analysis into the factors which led to the Qing downfall, it suffers from a similar limitation effecting all treaties and legal documents. This focuses on the fact that the agreements outlined in the document may not be non-distinct from the interpretation by the public of each empire, hence effecting the societal morale towards the dynasty. Due to the vast differences in the Chinese and English languages, it is clear that the Chinese translation, a product of the Qing society, was presented to the emperor in a heavily re-worded fashion, favouring China. While it maintained the core points of the agreements, the text suggests that the emperor is graciously providing the West with the aforementioned privileges and maintains absolutely no implication of a forceful signing. Hence, while this invaluable document explains the reality of the legal situation, it cannot be used as a descriptor for an analysis into the morale of the Chinese population toward the Qing: a rather significant element in their ultimate downfall.

\subsection{``The Condition of China'': A \textit{Times} Newspaper Column by the \textit{Special Correspondent} to China}

The following newspaper column was written by the \textit{Special Correspondent} to China for \textit{The Times}, published in August, 1884. The column outlines the seldom amount of hope that is left for the reduction and eventual abolition of opium smoking within China, stating that Parliamentary-intervention would be meaningless, and any changes must be invoked by the Chinese population-at-large.

\begin{fancyquote}
	[\ldots] The Chinaman, like men of other races, insists upon indulging in some stimulant or narcotic, and he has chosen opium. He is by no means the teetotaller which he is credited to be. Temperance societies exist in China. Still the Chinaman generally does not indulge in beer or wine --- a great debarrent being the cost when delivered from Europe --- and his \textit{samshu}\footnote{\textit{samshu}, mainly known as \textit{baiju} to natives (\jap{白酒} in Chinese), is a traditional Chinese spirit that is extremely strong at approximately 90 percent proof
	\autocite{Antkiewicz:1993}.} is a weakly subterfuge. The vice which it pleases him to indulge in is, therefore, opium. We have not yet succeeded in introducing temperance, far less abstinence, into England. And you may as soon expect the average Briton to give up his beer or spirits as the Chinaman his pipe. In neither case can you make man moral by Act of Parliament. No reform, I feel certain, is likely to come from the mandarinate, who are nearly without exception slaves to the habit, while the few free from it are powerless against it. It must come, if ever it does, by social reform, from the people themselves. The import of Indian opium by our Government is said by the missionaries to create a source of considerable difficulty in their relations with the Chinese. If not altogether a sincere belief with Chinamen, it is at least a highly convenient argument, and much used by them, that we are largely responsible for the prevalence of the habit; and not only the officials and \textit{literati} [the far-Eastern intelligentsia], but not a few foreigners even, have done their best to foster the idea. True or not true, the charge is one difficult to meet so long as Government preserves its present attitude with regard to Indian opium. Having in view the facts brought forward in this letter --- though of opinion that the suppression of the Indian opium traffic will not stop nor even diminish its use --- I think Government should take steps to discontinue it, and replace it by some other means of revenue. It can hardly be called a creditable source of revenue. From the practical financial as apart from the moral or sentimental aspect, it is advisable to examine the question and seek some means of replacing it. If no such steps be taken, Government will have lost the opportunity of carrying a measure of progress --- an act of self-respect as well as expediency --- for the import of Indian opium into China is doomed. [\ldots]
	\begin{flushright}
		\autocite{SpecialCorrespondent:1884}
	\end{flushright}
\end{fancyquote}


In the column, the writer illustrates the extent to which opium addiction is widespread throughout China by making a comparison to the entire British population achieving abstinence to beer and wine. This renders this source extremely useful due not only to its unsettling content, but also due to the high and established status of the writer.

Written over twenty years since the conclusion of the \textit{Second Opium War}\footnotemark, this article explains how morality cannot be achieved by mere legislation by either nation, but instead must come from the desires of the population. While much of the column is spent discussing the trade from a financial and economics point-of-view, such that the writer explains why the revenue generated from the trade is incredibly unreliable and unsustainable, the points raised regarding the social implications greatly compliment the usefulness of the document when analysing the reasons of the Qing dynasty's demise.

\footnotetext{The \textit{Second Opium War} was fought just over four years from \nth{8} October 1856, to \nth{24} October 1860 and had a subject matter rather unrelated to the opium trade. It was initially caused by Chinese officials boarding the Hong Kong/British ship, the \textit{Arrow}, and removing twelve Chinese men
\autocite{Wong:2002}; this is now known as the \textit{Arrow incident}
\autocite{Wong:1974}. Eventually resolved by the \textit{Treaty of Tientsin}, in which concessions were granted to the Western nations of Russia, France, the United Kingdom, and the United States
\autocite{Nield:2015}, the so-called ``unequal treaty'' has been described as a continuation to the \textit{Treaty of Nanjing}
\autocite{Wang:2008}. In particular, Britain were granted rights to station representatives is Beijing
\autocite{HKPress:1912}, even gaining access to the \textit{Forbidden City}
\autocite{Dong:2004} in addition to the revocation of all anti-religious Qing laws
\autocite{Chassiron:1861}. Russia were also granted the rights to use maritime trading ports
\autocite{Adamov:1952}.

Knowledge of the \textit{Treaty of Tientsin} is vital when analysing post-1860 documents, as the ``China proper'' was no longer under the effective control of the Qing, but instead the various aforementioned Western nations.}

Whilst the status and identity of the \textit{Special Correspondent}\footnote{The true identity of the \textit{Special Correspondent} has been investigated via personal correspondence with \textit{The Times} reference library, however archival records have proven inconclusive. It is rather safe to assume, by extrapolating from the identities of other contemporary foreign correspondents and inspection of the language used in the article, that the particulars of the \textit{Special Correspondent} are that of a well-educated British-born columnist.} is undoubtedly instrumental when analysing the accuracy and intentions of the document, the content of the column reveals a rather anti-Anglo stance seen by the description and personification of opium as something to be ``free from'' and ``powerless against''.

%Also raised is the point regarding the missionaries' opinions regarding the extent to which the trade severs any potential relations between Britain and the Qing, with some scholars citing that despite the vast differences in the cultural aspects of the two empires, a positive relationship is necessary for bilateral continued success succeeding the series of \textit{Unequal Treaties}
%\autocite{Fairbank:1942}.

\pagebreak
\printbibliography[title={Cited Works}, heading=bibintoc]

\end{document}
